% Paket für Deutsche Sprache (Übersetzungen von Chapter zu Kapitel, 
% richtige Umlaute, richtige % Silbentrennung)
% siehe auch http://de.wikipedia.org/wiki/Babel-System
\usepackage[english]{babel}

% Eingabecodierung, deutsche Umlaute oder die akzentuierten Zeichen sind verfügbar 
% und können direkt eingegeben werden
% siehe auch http://de.wikipedia.org/wiki/UTF8
\usepackage[utf8]{inputenc}

% Ausgabeschriftart von LaTeX festlegen
% siehe auch http://de.wikibooks.org/wiki/LaTeX-Schnellkurs:_Erste_Schritte
\usepackage[T1]{fontenc}

% Standardpfad für Grafiken
\graphicspath{{logos/}{bilder/}}

% Paket zum Erstellen von Plots mit TikZ
\usepackage{pgfplots}
% immer die neueste Version benutzen
\pgfplotsset{compat=newest}
% Verbindung von Linien durch eine schräge Kante
\pgfplotsset{every axis/.append style={line join=bevel}}
% Formatvorlage für Präsentationen
\mode<beamer>{
	\pgfplotsset{
		beamer/.style={
			width=0.8\textwidth,
			height=0.45\textwidth,
			legend style={font=\scriptsize},
			tick label style={font=\footnotesize},
			label style={font=\small},
			max space between ticks=28,
		}
	}
}
\mode<handout>{
	\pgfplotsset{
		beamer/.style={
			width=0.8\textwidth,
			height=0.45\textwidth,
			legend style={font=\scriptsize},
			tick label style={font=\footnotesize},
			label style={font=\small},
			max space between ticks=25,
		}
	}
}
\mode<article>{
	\pgfplotsset{
		beamer/.style={
			width=0.8\textwidth,
			height=0.45\textwidth,
			max space between ticks=35,
		}
	}
}
% neue Größenvorlage für zwei Plots nebeneinander anlegen
\pgfplotsset{
	scriptsize/.style={
		width=0.34\textwidth,
		height=0.1768\textwidth,
		legend style={font=\scriptsize},
		tick label style={font=\scriptsize},
		label style={font=\footnotesize},
		title style={font=\footnotesize},
		every axis title shift=0pt,
		max space between ticks=25,
		every mark/.append style={mark size=7},
		major tick length=0.1cm,
		minor tick length=0.066cm,
	}
}
\pgfplotsset{
	small/.style={
		width=6.5cm,
		height=,
		tick label style={font=\footnotesize},
		label style={font=\small},
		legend style={font=\footnotesize},
		max space between ticks=30,
	}
}
% Legendeneintrage standardmäßig links ausrichten
\pgfplotsset{legend cell align=left}
% Hauptgitternetz zeichnen
\pgfplotsset{xmajorgrids}
\pgfplotsset{ymajorgrids}
% Anzahl der kleinen Teilstriche zwischen zwei großen Teilstrichen
%\pgfplotsset{minor x tick num={3}}
%\pgfplotsset{minor y tick num={3}}
% feines Gitternetz zeichnen
%\pgfplotsset{xminorgrids}
%\pgfplotsset{yminorgrids}
% nur nach den Achsen skalieren
\pgfplotsset{scale only axis}
% Farben wie in MATLAB definieren
\definecolor{matlab1}{rgb}{0,0,1}
\definecolor{matlab2}{rgb}{0,0.5,0}
\definecolor{matlab3}{rgb}{1,0,0}
\definecolor{matlab4}{rgb}{0,0.75,0.75}
\definecolor{matlab5}{rgb}{0.75,0,0.75}
\definecolor{matlab6}{rgb}{0.75,0.75,0}
\definecolor{matlab7}{rgb}{0.25,0.25,0.25}
% Farbreihenfolge wie in MATLAB definieren
\pgfplotscreateplotcyclelist{matlab}{
	{matlab1,solid},
	{matlab2,dashed},
	{matlab3,dashdotted},
	{matlab4,dotted},
	{matlab5,densely dashed},
	{matlab6,densely dashdotted},
	{matlab7,densely dotted}% dies unterdrückt einen Fehler
}
% Farbreihenfolge wie in MATLAB benutzen
\pgfplotsset{cycle list name=matlab}
% Farbreihenfolge von pgfplots benutzen
%\pgfplotsset{cycle list name=color list}
% nur Graustufen benutzen
%\pgfplotsset{cycle list name=linestyles}
% Strichstärke auf 1pt festlegen
\pgfplotsset{every axis plot/.append style={line width=1pt}}
% für deutsche Dokumente ein Komma benutzen
\addto\extrasngerman{\pgfplotsset{/pgf/number format/.cd,set decimal separator={{{,}}}}}
% ein halbes Leerzeichen als Tausendertrennzeichen benutzen
%\pgfplotsset{/pgf/number format/.cd,1000 sep={\,}}
% kein Tausendertrennzeichen verwenden
\pgfplotsset{/pgf/number format/.cd,1000 sep={}}
% Zahlen kleiner als 0.1 auch im fixed-Format ausgeben
\pgfplotsset{/pgf/number format/.cd,std=-2}
% neue Positionen für Legenden anlegen
\pgfplotsset{/pgfplots/legend pos/north/.style={/pgfplots/legend style={at={(0.50,0.97)},anchor=north}}}
\pgfplotsset{/pgfplots/legend pos/south/.style={/pgfplots/legend style={at={(0.50,0.03)},anchor=south}}}
\pgfplotsset{/pgfplots/legend pos/east/.style={/pgfplots/legend style={at={(0.97,0.50)},anchor=east}}}
\pgfplotsset{/pgfplots/legend pos/west/.style={/pgfplots/legend style={at={(0.03,0.50)},anchor=west}}}
\pgfplotsset{/pgfplots/legend pos/outer north/.style={/pgfplots/legend style={at={(0.50,1.03)},anchor=south}}}

% Paket für SI-Einheiten
\usepackage[load-configurations=binary]{siunitx}
% Trennzeichen für Bereiche
\addto\extrasngerman{\sisetup{range-phrase={ bis~}}} 
\addto\extrasenglish{\sisetup{range-phrase={ to~}}}

% Paket, um das Floating in Article-Modus abzuschalten
\usepackage{float}

% Paket, um anderen Zeilenabstand einzustellen, besonders für Tabellen
\usepackage{setspace}

% Paket für ein intelligentes Leerzeichen
\usepackage{xspace}

% Abkürzung für z. B.
\newcommand{\zB}{z.\,B.\xspace}

% Paket für schönere Brüche im Textmodus
\usepackage{xfrac}
% Standardeinstellung mit einem schrägen Bruchstrich
\UseCollection{xfrac}{plainmath}

% schönere Tabellen
\usepackage{booktabs}

% create notes
\usepackage{pgfpages}
\usepackage{palatino}

